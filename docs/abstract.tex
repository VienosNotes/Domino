\documentclass[14pt,dvipdfm,trans]{beamer}
% pdfの栞の字化けを防ぐ
% テーマ
\usetheme{CambridgeUS}
% navi. symbolsは目立たないが,dvipdfmxを使うと機能しないので非表示に
\setbeamertemplate{navigation symbols}{} 
\usepackage{graphicx}
\usepackage{amsmath}
\usepackage{amssymb}
% フォントはお好みで
\usepackage{txfonts}
\usepackage{strike}
\mathversion{bold}
\renewcommand{\familydefault}{\sfdefault}
\renewcommand{\kanjifamilydefault}{\gtdefault}
\setbeamerfont{title}{size=\large,series=\bfseries}
\setbeamerfont{frametitle}{size=\large,series=\bfseries}
\setbeamertemplate{frametitle}[default][center]
\usefonttheme{professionalfonts} 
%
\usepackage{listings}
\definecolor{dkgreen}{rgb}{0,0.6,0}
\definecolor{dkblue}{rgb}{0,0,0.6}
\definecolor{skyblue}{rgb}{0.3,0.3,1}
\definecolor{dkred}{rgb}{0.6,0,0}

\renewcommand{\lstlistingname}{リスト}
\lstset{%language=Perl,
  basicstyle=\ttfamily\scriptsize,
  commentstyle=\color{dkgreen},
  stringstyle=\color{dkred},
  classoffset=0,
  keywordstyle=\bfseries,
  frame=trbl,
  framesep=3pt,
  showstringspaces=false,
  numbers=left,
  stepnumber=1,
  numberstyle=\tiny,
  tabsize=1,
  morekeywords={grammar, token, rule, method, make,
  actions, if, elsif, else, given, when, my},
  keywordstyle=\color{dkblue},
  classoffset=1,
  morekeywords={Str, ast, say, parse, eq},
  keywordstyle=\color{skyblue},
  classoffset=0,
  xleftmargin=2zw,
  xrightmargin=2zw,
  morecomment=[l]{\#},
  morestring=[s][\color{orange}]{<}{>},
  morestring=[s]{"}{"},
}


\title{Perl6の正規表現(+α)でLispモドキを実装}
\author{青木大祐}
\institute{筑波大学 情報科学類}
\date{2012年1月29日}

\usepackage{color}	% required for `\textcolor' (yatex added)
\begin{document}
\frame{\titlepage}
\begin{frame}
 \frametitle{このLTの概要}
 \vspace{2zh}
 \begin{itemize}
  \item Perl6 Grammarの紹介
	\vspace{3zh}
  \item Grammarを使って遊ぶ       
 \end{itemize}
 \vspace{2zh}
 \end{frame}

\begin{frame}
 \frametitle{Grammarとは?}
 \begin{itemize}
  \item <1-> 一般的には解析表現文法\\(\it{Parsing Expression Grammar:\textcolor{red}{PEG}})と呼ぶらしい
  \item <2-> 小さい名前付きルールから、順に組み上げて行く構造的な表現
  \item <3-> 正規表現と併用する事でさらに強力な記述が可能
 \end{itemize}
\end{frame}

\begin{frame}
\frametitle{Grammarで何が出来るのか}
普通の正規表現と違って…
\vspace{1zh}
\begin{itemize}
 \item <2-> 構造的な文字列を表現しやすい
 \item <3-> 再帰的な構造を記述できる

       \begin{itemize}
	\item <4-> \textcolor{red}{普通の}正規表現では再帰構造を記述できない
	\item <5-> PCRE(Perl Compatible Regular Expressions)系の\\
	      正規表現(?)エンジンなら可能
       \end{itemize}
\end{itemize}
\pause
\vspace{2zh}
\hspace{2zh}
\uncover<6->{\large{\textcolor{red}{→パーサとか書くのが楽になるかも}}}
\end{frame}

\begin{frame}[fragile]
 \frametitle{イメージとしてはこんな感じ}
 \begin{lstlisting}
grammar Lisp {
    token num { \d+ }
    token str { '"'\w+'"' }
    token literal { <num>|<str> }
    token nil { 'nil'|'(' <.ws> ')' }
    token symbol { \w+ }
    token atom { <literal>|<nil>|<symbol> }

    rule sexpr { <atom> | '\''? '(' <sexpr>+? % <.ws> ')' }
    rule TOP { ^^ <sexpr> $$ }
}
 \end{lstlisting} 
\pause
\begin{itemize}
 \item <2-> 「リテラル」は「数字」と「文字列」
 \item <3-> 「アトム」は「リテラル」と「nil」と「シンボル」
 \vspace{0.2zh}
 \item <4-> 「S式(S-Expressions)」は「アトム」と\\「S式のリスト」
 \vspace{0.5zh}
       \begin{itemize}
	\item <5-> ↑再帰構造
       \end{itemize}
\end{itemize}

\end{frame}

\begin{frame}[fragile]
 \frametitle{実際にパースしてみる [1]}
使い方は簡単:\\
\hspace{1zh}parseメソッドに対象の文字列を投げるだけ\\
\hspace{1zh}→受理されればMatchオブジェクトが返ってくる
\vspace{3zh}
\begin{lstlisting}
my $str = '(equal (car '(1 2)) 2)';
my $m = Lisp.parse($str);
\end{lstlisting}
\end{frame}

\begin{frame}[fragile]
 \frametitle{実際にパースしてみる [2]}
\begin{lstlisting}
=> <(equal (car '(1 2)) 1)>
 sexpr => <(equal (car '(1 2)) 1)>
  sexpr => <equal >
   atom => <equal>
    symbol => <equal>
  sexpr => <(car '(1 2)) >
   sexpr => <car >
    atom => <car>
     symbol => <car>
   sexpr => <'(1 2)>
    sexpr => <1 >
     atom => <1>
      literal => <1>
       num => <1>
    sexpr => <2>
     atom => <2>
      literal => <2>
       num => <2>
  sexpr => <1>
   atom => <1>
    literal => <1>
     num => <1>
\end{lstlisting}
\end{frame}

\begin{frame}
 \frametitle{更に一歩}
\begin{itemize}
 \item <1-> パースするだけじゃ勿体無い
 \vspace{0.5zh}
 \item <2-> 手動でMatchオブジェクトを走査するのは面倒
       \begin{itemize}
	\item <3-> トークンに刻むだけなら正規表現でも(ry
       \end{itemize}
\end{itemize}
\pause
\vspace{2zh}
\hspace{2zh}\uncover<4->{\large{→Actionを使ってスマートに処理}}
\end{frame}

\begin{frame}[fragile]
 \frametitle{Actionと抽象構文木 [1]}
parseメソッドはactionsオプションを指定できる
\vspace{1zh}
\begin{lstlisting}
 my $m = Lisp.parse($str, actions => SomeAction);
\end{lstlisting}
\pause
\begin{itemize}
 \item <2-> パースする際と同様に\textcolor{red}{深さ優先探索}でマッチン
       グを試みる
 \item <3-> トークンへのマッチが\textcolor{red}{成功}するたびに、actions
       に指定したクラスの同名メソッドが呼ばれる
       \vspace{1zh}
       \begin{itemize}
	\item <4-> actionは深さ優先探索の戻り掛けに実行される
	      \vspace{0.5zh}
	\item <5-> 最終的にマッチングに失敗するとしても、成功した所まで
	      はactionが実行される点に注意
       \end{itemize}
\end{itemize}
\end{frame}

\begin{frame}
 \frametitle{Actionと抽象構文木 [2]}
 \begin{itemize}
  \item <1-> 引数として、マッチに成功したトークンのMatchオブジェクトが渡
	される
	\vspace{0.5zh}
  \item	<2-> Matchオブジェクトを使って好きな処理をした後、抽象構文木
	(\it{Abstract Syntax Tree:\textcolor{red}{AST}})を作る
	\begin{itemize}
	 \item <3-> returnで値を返すのと同じように、make文を使う
	\end{itemize}
	\vspace{0.3zh}
  \item <4-> 作られた抽象構文木は、それを含む(つまり上の階層の)Matchオブ
	ジェクトから参照できる(\$m.ast)
 \end{itemize}
 \vspace{1zh}
 \pause
 \uncover<5->{→下位要素の抽象構文木を参照しながら、上(TOP)に向かって抽象構文
 木を組み立てて行く}
\end{frame}

\begin{frame}
\frametitle{Grammarで遊んでみる}
\hspace{0.5zh}目標:
\begin{itemize}
 \item <1-> \textcolor{red}{\tt{(equal (car '(1 2)) 2)}}を評価する
\vspace*{0.3zh}
 \item <2-> 関数はequalとcarとquoteが評価できれば良い
       \vspace*{0.5zh}
       \begin{itemize}
	\item <3-> equal: 2つの引数が等しいなら\textcolor{red}{t}、そうでないなら\textcolor{red}{nil}
	\item <4-> car: 引数に受け取ったリストの先頭の要素を返す
	\item <5-> quote: 引数に受け取ったS式をそのまま返す 
       \end{itemize}
       \vspace*{1zh}
 \item <6-> 正しく評価できていれば\textcolor{red}{nil}が返るはず
\end{itemize} 
\end{frame}

\begin{frame}[fragile]
 \frametitle{実際のActionのコード}
 \begin{lstlisting}
class Evaluate {
    method num ($/) { make $/.Str }
    method str ($/) { make $/.Str }
    method literal ($/) { make $<num>.?ast // $<str>.ast }
    method nil ($/) { make 'nil' }
    method symbol ($/) { make $/.Str }
    method atom ($/) { 
      make $<literal>.?ast // $<nil>.?ast // $<symbol>.ast 
    }
    ...
 \end{lstlisting}
 \begin{center}
  \begin{tabular}{ ccc }
   \uncover<2->{\textcolor{dkgreen}{num}と\textcolor{dkgreen}{str}のast &
   → & 文字列そのまま}\\
   \hline \uncover<3->{\textcolor{dkgreen}{literal}のast & → &
	   \textcolor{dkgreen}{num}か\textcolor{dkgreen}{str}のast}\\
   \hline \uncover<4->{\textcolor{dkgreen}{nil}のast & → & \textcolor{brown}{'nil'}}\\
   \hline \uncover<5->{\textcolor{dkgreen}{symbol}のast & → & 文字列そのまま}\\
   \hline \uncover<6->{\textcolor{dkgreen}{atom}のast & → & 中身のast}\\
  \end{tabular}  
 \end{center}
\end{frame}

\begin{frame}[fragile]
 \frametitle{S式の解釈 [1]}
\begin{lstlisting}
method sexpr ($/) {
  # quoted expressions
  if $/.Str.match(/^^ \'/) {
     make $/.Str.substr(1);
  }
  # atoms
  elsif $<atom> {
    make $<atom>.ast;
  }
  # other expressions
  else {
    given $<sexpr>[0].ast {
      when "car" {
        make $<sexpr>[1]<sexpr>[0].ast;
      }
      when "equal" {
        make $<sexpr>[1].ast eq $<sexpr>[2].ast ?? "t" !! "nil";
      }
    }
  }
}
\end{lstlisting}
\end{frame}

\begin{frame}[fragile]
 \frametitle{S式の解釈 [2]}
 \begin{lstlisting}[firstnumber=2]
  # quoted expressions
  if $/.Str.match(/^^ \'/) {
     make $/.Str.substr(1);
  } 
 \end{lstlisting}
\begin{itemize}
 \item <2-> 先頭がクォートから始まるS式(quote関数)
       \begin{itemize}
	\item <3-> 引数に受け取ったS式をそのまま返す 
	\item <4-> クォートを外してmake
       \end{itemize}
\end{itemize}
 \pause[5]
 \begin{lstlisting}[firstnumber=6]
   # atoms
  elsif $<atom> {
    make $<atom>.ast;
  }
 \end{lstlisting}
\begin{itemize}
 \item <6-> atomから成るS式
 \begin{itemize}
   \item <7-> atomのastをそのまま使う
 \end{itemize}

\end{itemize}
\end{frame}
 
\begin{frame}[fragile]
 \frametitle{S式の解釈 [3]}
\begin{lstlisting}[firstnumber=10]
  # other expressions
  else {
    given $<sexpr>[0].ast {
      when "car" {
        make $<sexpr>[1]<sexpr>[0].ast;
      }
\end{lstlisting}
\uncover<2->{その他の式: S式の1番目の要素で分岐}
\vspace*{1zh}
\begin{itemize}
 \item <3-> car関数

       \begin{itemize}
	\item <4-> 引数に受け取ったリストの先頭の要素を返す
	\item <5-> 2番目のS式(つまり引数に受け取ったS式)の\\
	      1番目の要素のastを返す
       \end{itemize}
\end{itemize}
\end{frame}

\begin{frame}[fragile]
\frametitle{S式の解釈 [4]}
\begin{lstlisting}[firstnumber=16]
       when "equal" {
        make $<sexpr>[1].ast eq $<sexpr>[2].ast ?? "t" !! "nil";
      }
    }
  }
}
\end{lstlisting}
\begin{itemize}
 \item <2-> equal関数
       \vspace*{0.5zh}
       \begin{itemize}
	\item <3-> 2つの引数が等しいならt、そうでないならnil
	\item <4-> 2番目のS式(つまり第1引数)のastと\\
	      3番目のS式(つまり第2引数)のastを比較して、\\
	      等しいならtを、そうでないならnilを返す
       \end{itemize}
\end{itemize}
\end{frame}

\begin{frame}[fragile]
 \frametitle{実行してみる}
実行部分:\\
\pause
\hspace*{1zh}
コマンドライン引数に評価したい式を渡す
\begin{lstlisting}
my $str = @*ARGS[0];
my $m = Lisp.parse($str, actions => Evaluate);
say $m.ast;
\end{lstlisting}
\pause
先程の式を渡すと…
\begin{lstlisting}[numbers=none]
% perl6 lisp.pl "(equal (car '(1 2)) 2)"
nil
\end{lstlisting}
\pause
少し変えてみる
\begin{lstlisting}[numbers=none]
% perl6 lisp.pl "(equal (car '(1 2)) 1)"
t
\end{lstlisting}
\pause
\begin{center}
\textcolor{red}{\large{ちゃんと動いてるっぽい!}}
\end{center}
\end{frame}

\begin{frame}
 \frametitle{喜ぶのはまだ早い?}
\begin{center}
 目標の要求定義が適当だったので見逃しているが…\\
 \uncover<2->{この設計には致命的な\textcolor{red}{欠陥}がある!}
\end{center} 
\pause
\begin{itemize}
 \item <3-> \textcolor<4->{gray}{quote: 引数に受け取ったS式をそのまま返す}
 \item <4-> quote: 引数に受け取ったS式を\textcolor{red}{評価せずに}返す
\end{itemize}
\end{frame}

\begin{frame}[fragile]
 \frametitle{Actionの実行される順番を考えてみる [1] }
\begin{lstlisting}
 => <(equal (car '(1 2)) 1)>          #22
 sexpr => <(equal (car '(1 2)) 1)>     #21
  sexpr => <equal > #3
   atom => <equal>   #2
    symbol => <equal> #1
  sexpr => <(car '(1 2)) >   #16
   sexpr => <car > #6         \
    atom => <car>   #5         \
     symbol => <car> #4         \
   sexpr => <'(1 2)>             #15
    sexpr => <1 >  #10
     atom => <1>    #9
      literal => <1> #8
       num => <1>     #7
    sexpr => <2>   #14
     atom => <2>    #13
      literal => <2> #12
       num => <2>     #11
  sexpr => <1>                        #20
   atom => <1>                         #19
    literal => <1>                      #18
     num => <1>                          #17
\end{lstlisting}
\end{frame}

\begin{frame}[fragile]
 \frametitle{Actionの実行される順番を考えてみる [2] }
\begin{lstlisting}
 => <(equal (car '(1 2)) 1)>          #22
 sexpr => <(equal (car '(1 2)) 1)>     #21
  sexpr => <equal > #3
   atom => <equal>   #2
    symbol => <equal> #1
  sexpr => <(car '(1 2)) >   #16
   sexpr => <car > #6         \
    atom => <car>   #5         \
     symbol => <car> #4         \
   sexpr => <'(1 2)>             #15
    sexpr => <1 >  #10 <== 評価している!
     atom => <1>    #9
      literal => <1> #8
       num => <1>     #7
    sexpr => <2>   #14 <== 評価している!
     atom => <2>    #13
      literal => <2> #12
       num => <2>     #11
  sexpr => <1>                        #20
   atom => <1>                         #19
    literal => <1>                      #18
     num => <1>                          #17
\end{lstlisting}
\end{frame}

\begin{frame}
 \frametitle{副作用の悪夢}
 もし\tt{'(1 2)}が\tt{'(\textcolor{red}{(print 1)} 2)}だったら、というお
 話\\
\pause
\begin{itemize}
 \item Lispには
\begin{itemize}
 \item 関数(Functions) \pause
 \item \textcolor{red}{特殊形式}(Special Forms)
       \begin{itemize}
	\item quote, if, cond, define, etc.
       \end{itemize}
\end{itemize}
\hspace*{12zh}がある
\vspace{1zh}
\pause[4]
 \item 関数では表現できない特別な評価ルール
       \begin{itemize}
	\item <5-> ex. (if cond (expr1) (expr2))
	\item <6-> expr1/expr2のどちらか\textcolor{red}{のみ}を評価したい
	\item <7-> 今の設計では両方評価してしまう
       \end{itemize}
\end{itemize}
\end{frame}

\begin{frame}
\frametitle{何故こうなるのか}
\begin{itemize}
 \item <2-> \textcolor{red}{深さ優先探索}の\textcolor{red}{戻り掛け}だか
       ら
 \vspace{1zh}
 \item <3-> つまりクォートされた式の内側に居るかどうかを知る手段が無い
\end{itemize}
\vspace{3zh}
\uncover<4->{\hspace{2zh}→やはり無理か?}
\end{frame}

\begin{frame}[fragile]
 \frametitle{解決策}
 \pause
 方針:特殊形式の中に入るS式を区別したい
 \vspace{1zh}
 \pause
 \begin{lstlisting}
 token spform { 'cond' | 'if' | 'define' | 'quote' }

 # sexpr with special forms
 rule sp_sexpr { '(' <spform> [<ne_sexpr>+? % <.ws>] ')' }
 # sexpr with no evaluate
 rule ne_sexpr { <atom> || '(' [<ne_sexpr>+? % <.ws>] ')' }
 \end{lstlisting}
\begin{itemize}
 \item <4-> 特殊形式の中は別のルール(ne\_sexpr)にわける
 \item <5-> S式と同じ構造だが、actionは何もせずに文字列を返すように実装
 \item <6-> sp\_sexprは元の文字列を使って上手くやる
\end{itemize}
\end{frame}

\begin{frame}
 \frametitle{まとめ}
 \pause
 \begin{itemize}
  \item <2-> Perl6はGrammarという機能を使って構造的な文字列を簡単にパー
	スできる
  \item <3-> Actionsを指定する事で、パースしながら抽象構文木を作れる
  \item <4-> Actionsの処理は制約が大きいので、それも混みでGrammarを書い
	た方が最終的に楽
 \end{itemize}
\end{frame}

\begin{frame}
\frametitle{おまけ}
 全体的に上手くやってるつもり(自称)の途中実装\\

 \uncover<2->{→ http://github.com/VienosNotes/Domino}
\vspace*{1zh}
\begin{itemize}
 \item <3-> 発表のコードはかなり簡略化したもの
 \item <4-> ちゃんと他の関数も使えるように
 \item <5-> できればPerl6の関数も呼べると良いよね
\end{itemize}
\vspace{3zh}
 \hspace{18zh} \uncover<6>{おしまい}
\end{frame}
\end{document}